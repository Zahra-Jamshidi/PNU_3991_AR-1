\documentclass{book}
\begin{document}
\begin{flushright}
\texttt{INTRODUCTION}
\hspace*{0.5cm}
\textbf{13}
\end{flushright}
\vspace*{0.7cm}
futuristic writings of its inventor, William Gibson, and less so in the more mundane
world of network-based education development and delivery. Likewise, we do not
aggrandize the notion of partnership between networked technology
and human beings to indulge in the language of cyborgs or cybernetics-even thought we remain
open to the notion of the continuous development of some quite astounding
technical aids to human processes, many of which will be neurologically linked directly to
our bodies.Although much of the context of networking focuses on communication
among network users, we also do not use the term \emph{computer mediated communication}.
The Net provides access to data, virtual environments, textbook, and many other
nonhuman reference sources. Describing the use of these resources as communications
seems too anthropomorphic for our linking. Thus, we are left with a shortage of precise
and well-understood terminology. We have settled on the use of the adjective \emph{networked}
and noun \emph{Net}(with a capital) to describe this context. Net seems to reflect
the technical nature of the environment, but also carries with it the context of human
interconnectedness that is critical to educational applications of the Internet.
\\\hspace*{0.5cm} Our discussion of terminology underscores the multiple functions of the Net.
At one level the Net is merely a technology, one that is based on digital transmission,
routing, error checking, and sending and receiving of data in many formats.
These transmission may be private and exclusive to as few as two participants or as wide as
broadcasts to millions. At the same time, the Net is a rich social environment or
context in which many aspects of human life schooling, to commerce. to sex, are supported.
The Net is also a sociological and psychological filter, in which ideas are
formatted and in many ways de-contextualized into text or audiovisual constructs.
The Net is also a business in which fortunes are made and lost. Finally the Net is a
repository, providing means and tools to store and retrieve a host of cultural. academic,
commercial, and technical data.
\\\hspace*{0.5cm}We have experienced even great difficulty describing non-networked activity,
which we often like to contrast to activity mediated via the network. Describing
non-networked research as "real" as opposed to "virtual" certainly does not work.
Describing all aspects of like that are not mediated on a networked as "offline" activity also seems
somewhat condescending and tehnocentric. We are also not comfortable with the somewhat
derogatory reference to humans as "wetware," "meatware," or "liveware."
Thus, when we are discussing non-networked activity or contexts we usually refer to
them as "face-to-face" or "traditional" and in their educational sense as " classroom" or
"campus-based."\\
\begin{flushleft}
\texttt{AN e-RESEARCH EXAMPLE}
\hspace*{0.5cm}
\end{flushleft}
\vspace*{0.5cm}
In the winter of 2001, Liam Rourke and I (Terry Anderson) developed a research proposal
to investigate the capacity and impact of peer moderators in the computer-mediated-communications
delivered graduate course that I was teaching. From our own teaching, we were aware
of the excessive time commitments involved in teaching online and of the
literature on peer teaching effectiveness. We had developed a tool to assess "teaching"
\end{document} 